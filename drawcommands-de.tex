\drawcmd{AlphaFunc}{Sets the alpha-test function}

\begin{framed}
\begin{verbatim}
void AlphaFunc(byte func,
               byte ref);
\end{verbatim}
\end{framed}

\begin{tabular}{lp{0.8 \textwidth}}

\\ \mach{func} & comparison function, one of
        \mach{NEVER}, \mach{LESS}, \mach{LEQUAL}, \mach{GREATER}, \mach{GEQUAL}, \mach{EQUAL}, \mach{NOTEQUAL}, or \mach{ALWAYS} \\

\\ \mach{ref} & reference value for comparison function \\

\end{tabular}

\vspace{10pt}
The alpha test function tests each pixel's
alpha value, and only draws the pixel
if the test passes.
For example
\begin{verbatim}
  GD.AlphaText(GEQUAL, 160);
\end{verbatim}
means that only pixels with $(A \ge 160)$ are drawn.  
The default state is \mach{ALWAYS}, which means that pixels are always drawn.
\minipng{0032}
\eg{examples-alphafunc}

\drawcmd{Begin}{Selects the graphics primitive for drawing}

\begin{framed}
\begin{verbatim}
void Begin(byte prim);
\end{verbatim}
\end{framed}

\begin{tabular}{lp{0.8 \textwidth}}

\\ \mach{prim} & one of
\mach{BITMAPS},
\mach{POINTS},
\mach{LINES},
\mach{LINE\_STRIP},
\mach{EDGE\_STRIP\_R},
\mach{EDGE\_STRIP\_L},
\mach{EDGE\_STRIP\_A},
\mach{EDGE\_STRIP\_B} or
\mach{RECTS} \\
\end{tabular}

\vspace{10pt}
The \mach{Begin} command sets the current graphics draw primitve.
It does not draw anything - that is done by a later \nameref{Vertex2f} or \nameref{Vertex2ii} command.
The drawing primitive can be:

\vspace{10pt}

\begin{tabular}{lp{0.8 \textwidth}}
\mach{BITMAPS     } & each vertex draws a bitmap \\
\mach{POINTS      } & each vertex draws an anti-aliased point  \\
\mach{LINES       } & each pair of vertices draws an anti-aliased line \\
\mach{RECTS       } & each pair of vertices draws an anti-aliased rectangle \\
\mach{LINE\_STRIP  } & the vertices define a connected line segment \\
\mach{EDGE\_STRIP\_A} & like \mach{LINE\_STRIP}, but draws pixels above the line \\
\mach{EDGE\_STRIP\_B} & like \mach{LINE\_STRIP}, but draws pixels below the line \\
\mach{EDGE\_STRIP\_L} & like \mach{LINE\_STRIP}, but draws pixels left of the line \\
\mach{EDGE\_STRIP\_R} & like \mach{LINE\_STRIP}, but draws pixels right of the line
\end{tabular}

\png{0047}

\drawcmd{BitmapHandle}{Sets the bitmap handle}

\begin{framed}
\begin{verbatim}
void BitmapHandle(byte handle);
\end{verbatim}
\end{framed}

\begin{tabular}{lp{0.8 \textwidth}}

\\ \mach{handle} & integer handle number, 0-15 \\

\end{tabular}

\vspace{10pt}
The \mach{BitmapHandle} command sets the current bitmap handle, used by
\nameref{Vertex2f},
\nameref{BitmapSource},
\nameref{BitmapLayout} and
\nameref{BitmapSize}.

The bitmap handle is part of the graphics context; its default value is 0.



\drawcmd{BitmapLayout}{Sets the bitmap layout}

\begin{framed}
\begin{verbatim}
void BitmapLayout(byte format,
                  uint16_t linestride,
                  uint16_t height);
\end{verbatim}
\end{framed}

\begin{tabular}{lp{0.8 \textwidth}}

\\ \mach{format} & pixel format of the bitmap, one of:
\mach{ARGB1555},
\mach{L1},
\mach{L4},
\mach{L8},
\mach{RGB332},
\mach{ARGB2},
\mach{ARGB4},
\mach{RGB565},
\mach{PALETTED},
\mach{TEXT8X8},
\mach{TEXTVGA},
\mach{BARGRAPH}.
\\

\\ \mach{linestride} & the size in bytes of one line of the bitmap in memory \\

\\ \mach{height} & height of the bitmap in pixels \\

\end{tabular}

\vspace{10pt}
The \mach{BitmapLayout} command sets the current bitmap's layout in memory.
The \mach{format} controls how memory data is converted into pixels. 
Each pixel in memory is 1,4,8 or 16 bits. The color is extracted from these bits as follows,
where ``v'' is the pixel data.

\vspace{10pt}
\begin{tabular}{cccccc}
format           & bits per & alpha        & red   & green & blue \\
                 &  pixel   &              &       &       &      \\
\hline
\mach{L1}        &    1     & v$_0$        &  1.0  &  1.0  &  1.0 \\
\mach{L4}        &    4     & v$_{3..0}$   &  1.0  &  1.0  &  1.0 \\
\mach{L8}        &    8     & v$_{7..0}$   &  1.0  &  1.0  &  1.0 \\
\hline
\mach{RGB332}    &    8     & 1.0          &  v$_{7..5}$   &  v$_{4..2}$  &  v$_{1..0}$ \\
\mach{RGB565}    &    16    & 1.0          &  v$_{15..1}$  &  v$_{10..5}$ &  v$_{4..0}$ \\
\hline
\mach{ARGB2}     &    8     & v$_{7..6}$   &  v$_{5..4}$   &  v$_{3..2}$  &  v$_{1..0}$ \\
\mach{ARGB4}     &    16    & v$_{15..12}$ &  v$_{11..8}$  &  v$_{7..4}$  &  v$_{3..0}$ \\
\end{tabular}
\vspace{10pt}

1 and 4 bit pixels are packed in bytes from left to right, so
leftmost pixels in the bitmap occupy the most significant bits in a byte.
16 bit pixels are little-endian in graphics memory, and must be aligned on even memory boundaries.

\mach{PALETTED} format uses 8 bits per pixel, each pixel is an index
into the 256 entry 32-bit color table loaded at \mach{RAM\_PALETTE}.
% \mach{TEXT8X8} interpret each byte in the bitmap as an index into a character array, with a character size of 8x8 pixels.
% \mach{TEXTVGA} interprets memory in the same format as an IBM PC VGA graphic controller, 8 bits of character attribute and an 8-bit character code. The character size is 8x16.


\drawcmd{BitmapSize}{Sets the bitmap size and appearance}

\begin{framed}
\begin{verbatim}
void BitmapSize(byte filter,
                byte wrapx,
                byte wrapy,
                uint16_t width,
                uint16_t height);
\end{verbatim}
\end{framed}

\begin{tabular}{lp{0.8 \textwidth}}

\\ \mach{filter} & bitmap pixel filtering, \mach{NEAREST} or \mach{BILINEAR} \\

\\ \mach{wrapx} & $x$ wrapping mode, \mach{BORDER} or \mach{REPEAT} \\

\\ \mach{wrapy} & $y$ wrapping mode, \mach{BORDER} or \mach{REPEAT} \\

\\ \mach{width} & on-screen drawn width, in pixels \\

\\ \mach{height} & on-screen drawn height, in pixels \\

\end{tabular}

\vspace{10pt}
The \mach{BitmapSize} command controls how the current bitmap appears on screen.



\drawcmd{BitmapSource}{Sets the bitmap source address}

\begin{framed}
\begin{verbatim}
void BitmapSource(uint32_t addr);
\end{verbatim}
\end{framed}

\begin{tabular}{lp{0.8 \textwidth}}

\\ \mach{addr} & base address for bitmap \mach{0x00000} - \mach{0x3ffff} \\

\end{tabular}

\vspace{10pt}
The \mach{BitmapSource} command sets the base address for the bitmap.
For 16-bit bitmaps, this address must be even.



\drawcmd{BlendFunc}{Sets the color blend function}

\begin{framed}
\begin{verbatim}
void BlendFunc(byte src,
               byte dst);
\end{verbatim}
\end{framed}

\begin{tabular}{lp{0.8 \textwidth}}

\\ \mach{src} & source blend factor, one of
\mach{ZERO},
\mach{ONE},
\mach{SRC\_ALPHA},
\mach{DST\_ALPHA},
\mach{ONE\_MINUS\_SRC\_ALPHA},
\mach{ONE\_MINUS\_DST\_ALPHA}.  \\

\\ \mach{dst} & destination blend factor, one of
\mach{ZERO},
\mach{ONE},
\mach{SRC\_ALPHA},
\mach{DST\_ALPHA},
\mach{ONE\_MINUS\_SRC\_ALPHA},
\mach{ONE\_MINUS\_DST\_ALPHA}.  \\

\end{tabular}

\vspace{10pt}
The \mach{BlendFunc} command sets the blend function used to combine pixels
with the contents of the frame buffer.
Each incoming pixel's color is multiplied by the source blend factor, and each frame buffer pixel
is multiplied by the destination blend factor.
These two results are added to give the final pixel color.

\minipng{0031}
\eg{examples-blendfunc}


% \drawcmd{Call}{Something}
% 
% \begin{framed}
% \begin{verbatim}
% void Call(uint16_t dest);
% \end{verbatim}
% \end{framed}
% 
% \begin{tabular}{lp{0.8 \textwidth}}
% 
% \\ \mach{dest} & something \\
% 
% \end{tabular}
% 
% \vspace{10pt}
% The \mach{Call} command

\drawcmd{Cell}{Sets the bitmap cell}

\begin{framed}
\begin{verbatim}
void Cell(byte cell);
\end{verbatim}
\end{framed}

\begin{tabular}{lp{0.8 \textwidth}}

\\ \mach{cell} & cell number 0-127 \\

\end{tabular}

\vspace{10pt}
The \mach{Cell} command sets the current bitmap cell used by the \nameref{Vertex2f} command.

\png{0034}
\eg{examples-cell}


\drawcmd{ClearColorA}{Sets the alpha component of the clear color}

\begin{framed}
\begin{verbatim}
void ClearColorA(byte alpha);
\end{verbatim}
\end{framed}

\begin{tabular}{lp{0.8 \textwidth}}

\\ \mach{alpha} & Clear color alpha component, 0-255 \\

\end{tabular}

\vspace{10pt}
The \mach{ClearColorA} command sets the clear color A channel value.
A subsequent \nameref{Clear} writes this value to the frame buffer alpha channel.



\drawcmd{Clear}{Clears the screen}

\begin{framed}
\begin{verbatim}
void Clear(byte c = 1,
           byte s = 1,
           byte t = 1);
\end{verbatim}
\end{framed}

\begin{tabular}{lp{0.8 \textwidth}}

\\ \mach{c} & if set, clear the color buffer \\

\\ \mach{s} & if set, clear the stencil buffer \\

\\ \mach{t} & if set, clear the tag buffer \\

\end{tabular}

\vspace{10pt}
The \mach{Clear} command clears the requested frame buffers.

\png{0035}
\eg{examples-clear}


\drawcmd{ClearColorRGB}{Sets the R,G,B components of the clear color}

\begin{framed}
\begin{verbatim}
void ClearColorRGB(byte red,
                   byte green,
                   byte blue);
void ClearColorRGB(uint32_t rgb);
\end{verbatim}
\end{framed}

\begin{tabular}{lp{0.8 \textwidth}}

\\ \mach{red} & red component 0-255 \\

\\ \mach{green} & green component 0-255 \\

\\ \mach{blue} & blue component 0-255 \\

\\ \mach{rgb} & 24-bit color in RGB order, \mach{0x000000}-\mach{0xffffff} \\

\end{tabular}

\vspace{10pt}
The \mach{ClearColorRGB} command sets the clear color R,G and B values.
A subsequent \nameref{Clear} writes this value to the frame buffer R,G and B channels.
\minipng{0030}
\eg{examples-clearcolorrgb}



\drawcmd{ClearStencil}{Sets the stencil clear value}

\begin{framed}
\begin{verbatim}
void ClearStencil(byte s);
\end{verbatim}
\end{framed}

\begin{tabular}{lp{0.8 \textwidth}}

\\ \mach{s} & stencil buffer clear value 0-255 \\

\end{tabular}

\vspace{10pt}
The \mach{ClearStencil} command sets the stencil buffer clear value.
A subsequent \nameref{Clear} writes this value to the stencil buffer.



\drawcmd{ClearTag}{Sets the tag clear value}

\begin{framed}
\begin{verbatim}
void ClearTag(byte s);
\end{verbatim}
\end{framed}

\begin{tabular}{lp{0.8 \textwidth}}

\\ \mach{s} & tag value 0-255 \\

\end{tabular}

\vspace{10pt}
The \mach{ClearTag} command sets the tag buffer clear value.
A subsequent \nameref{Clear} writes this value to the tag buffer.



\drawcmd{ColorA}{Sets the A component of the current color}

\begin{framed}
\begin{verbatim}
void ColorA(byte alpha);
\end{verbatim}
\end{framed}

\begin{tabular}{lp{0.8 \textwidth}}

\\ \mach{alpha} & alpha value 0-255 \\

\end{tabular}

\vspace{10pt}
The \mach{ColorA} command sets the alpha component of the current drawing color.
\png{0029}
\eg{examples-colora}

\drawcmd{ColorMask}{Sets the mask controlling color channel writes}

\begin{framed}
\begin{verbatim}
void ColorMask(byte r,
               byte g,
               byte b,
               byte a);
\end{verbatim}
\end{framed}

\begin{tabular}{lp{0.8 \textwidth}}

\\ \mach{r} & if set, enable writes to the red component \\

\\ \mach{g} & if set, enable writes to the green component \\

\\ \mach{b} & if set, enable writes to the blue component \\

\\ \mach{a} & if set, enable writes to the alpha component \\

\end{tabular}

\vspace{10pt}
The \mach{ColorMask} command sets the color mask, which enables color writes to the frame buffer
R,G,B and A components.

\minipng{0033}
\eg{examples-colormask}


\drawcmd{ColorRGB}{Sets the R,G,B components of the current color}

\begin{framed}
\begin{verbatim}
void ColorRGB(byte red,
              byte green,
              byte blue);

void ColorRGB(uint32_t rgb);
\end{verbatim}
\end{framed}

\begin{tabular}{lp{0.8 \textwidth}}

\\ \mach{red} & red component 0-255 \\

\\ \mach{green} & green component 0-255 \\

\\ \mach{blue} & blue component 0-255 \\

\\ \mach{rgb} & 24-bit color in RGB order, \mach{0x000000}-\mach{0xffffff} \\

\end{tabular}

\vspace{10pt}
The \mach{ColorRGB} command sets the current color.
This color is used by all drawing operations.

\minipng{0028}
\eg{examples-colorrgb}


\drawcmd{LineWidth}{Set the line width}

\begin{framed}
\begin{verbatim}
void LineWidth(uint16_t width);
\end{verbatim}
\end{framed}

\begin{tabular}{lp{0.8 \textwidth}}

\\ \mach{width} & line width in $1/16$th of a pixel \\

\end{tabular}

\vspace{10pt}
The \mach{LineWidth} command sets the line width used when drawing \mach{LINES} and \mach{LINE\_STRIP}.
The width is specified in $1/16$th of a pixel, so
\mach{LineWidth(16)} sets the width to 1 pixel.
Note that the width is the distance from the center of the line to its outside edge,
rather like the radius of a circle. So the total width of the line is double the value specified.

The maximum line width is 4095, or $255 \frac{15}{16}$ pixels.
\png{0026}
\eg{examples-linewidth}

\drawcmd{PointSize}{Set the point size}

\begin{framed}
\begin{verbatim}
void PointSize(uint16_t size);
\end{verbatim}
\end{framed}

\begin{tabular}{lp{0.8 \textwidth}}

\\ \mach{size} & point size in $1/16$th of a pixel \\

\end{tabular}

\vspace{10pt}
The \mach{PointSize} command sets the point size used when drawing \mach{POINTS}.
The size is specified in $1/16$th of a pixel, so
\mach{LineWidth(16)} sets the width to 1 pixel.
Note that the size is the distance from the center of the point to its outside edge,
that is, the radius. So the total width of the point is double the value specified.

The maximum point size is 4095, or $255 \frac{15}{16}$ pixels.
\png{0027}
\eg{examples-pointsize}


\drawcmd{RestoreContext}{Restore the drawing context to a previously saved state}

\begin{framed}
\begin{verbatim}
void RestoreContext(void);
\end{verbatim}
\end{framed}

\begin{tabular}{lp{0.8 \textwidth}}

\end{tabular}

\vspace{10pt}
The collected graphics state is called the \textit{graphics context}.
The \mach{SaveContext} command saves a copy of this state, and the \mach{RestoreContext} command restores this saved copy.

The hardware can preserve up to four graphics contexts in this way. The graphics context consists of:
\vspace{10pt}

\begin{tabular}{rl}
state & drawing commands \\
\hline
alpha-test function   &  \nameref{AlphaFunc} \\
bitmap handle         &  \nameref{BitmapHandle} \\
blend function        &  \nameref{BlendFunc} \\
bitmap cell           &  \nameref{Cell} \\
color clear value     & \nameref{ClearColorA}, \nameref{ClearColorRGB} \\
stencil clear value   & \nameref{ClearStencil} \\
tag clear value       & \nameref{ClearTag} \\
color write mask      & \nameref{ColorMask} \\
color                 & \nameref{ColorA}, \nameref{ColorRGB} \\
line width            & \nameref{LineWidth} \\
point size            & \nameref{PointSize} \\
scissor rectangle     & \nameref{ScissorSize}, \nameref{ScissorXY} \\
stencil test function & \nameref{StencilFunc} \\
stencil write mask    & \nameref{StencilMask} \\
stencil operation     & \nameref{StencilOp} \\
tag write mask        & \nameref{TagMask} \\
tag value             & \nameref{Tag}
\end{tabular}



\drawcmd{SaveContext}{Save the graphics context}

\begin{framed}
\begin{verbatim}
void SaveContext(void);
\end{verbatim}
\end{framed}

\begin{tabular}{lp{0.8 \textwidth}}

\end{tabular}

\vspace{10pt}
The collected graphics state is called the \textit{graphics context}.
The \mach{SaveContext} command saves a copy of this state, and the \mach{RestoreContext} command restores this saved copy.

\png{0036}
\eg{examples-context}

\drawcmd{ScissorSize}{Set the size of the scissor rectangle}

\begin{framed}
\begin{verbatim}
void ScissorSize(uint16_t width,
                 uint16_t height);
\end{verbatim}
\end{framed}

\begin{tabular}{lp{0.8 \textwidth}}

\\ \mach{width} & scissor rectangle width, in pixels, 0-512 \\

\\ \mach{height} & scissor rectangle height, in pixels, 0-512 \\

\end{tabular}

\vspace{10pt}
The \mach{ScissorSize} command sets the dimensions of the scissor rectangle.
The scissor rectangle limits drawing to a rectangular region on the screen.

\png{0037}
\eg{examples-scissor}


\drawcmd{ScissorXY}{Set the top-left corner of the scissor rectangle}

\begin{framed}
\begin{verbatim}
void ScissorXY(uint16_t x,
               uint16_t y);
\end{verbatim}
\end{framed}

\begin{tabular}{lp{0.8 \textwidth}}

\\ \mach{x} & $x$ coordinate of top-left corner of the scissor rectangle, 0-511 \\

\\ \mach{y} & $y$ coordinate of top-left corner of the scissor rectangle, 0-511 \\

\end{tabular}

\vspace{10pt}
The \mach{ScissorXY} command sets the top-left corner of the scissor rectangle.
The scissor rectangle limits drawing to a rectangular region on the screen.


\drawcmd{StencilFunc}{Set the stencil test function}

\begin{framed}
\begin{verbatim}
void StencilFunc(byte func,
                 byte ref,
                 byte mask);
\end{verbatim}
\end{framed}

\begin{tabular}{lp{0.8 \textwidth}}

\\ \mach{func} & set the stencil comparison operation, one of
        \mach{NEVER}, \mach{LESS}, \mach{LEQUAL}, \mach{GREATER}, \mach{GEQUAL}, \mach{EQUAL}, \mach{NOTEQUAL}, or \mach{ALWAYS} \\

\\ \mach{ref} & set the stencil reference value used for the comparison, 0-255 \\

\\ \mach{mask} & an 8-bit mask that is anded with both \mach{ref} and the pixel's stencil value before comparison, 0-255 \\

\end{tabular}

\vspace{10pt}
The \mach{StencilFunc} command controls the stencil testing operation.
During drawing, the stencil test is applied to each pixel, and if the test fails, the pixel is not drawn.
Setting \mach{func} to \mach{ALWAYS} means that pixels are always drawn.

\drawcmd{StencilMask}{Sets the mask controlling stencil writes}

\begin{framed}
\begin{verbatim}
void StencilMask(byte mask);
\end{verbatim}
\end{framed}

\begin{tabular}{lp{0.8 \textwidth}}

\\ \mach{mask} & Each set bit enables the corresponding bit write to the stencil buffer \\

\end{tabular}

\vspace{10pt}
The \mach{StencilMask} command controls writes to the stencil buffer.
Because the stencil buffer is 8 bits deep, each bit in \mach{mask} enables writes to the same bit of the stencil buffer.
So a \mach{mask} of \mach{0x00} disables stencil writes, and a \mach{mask} of \mach{0xff} enables stencil writes.


\drawcmd{StencilOp}{Set the stencil update operation}

\begin{framed}
\begin{verbatim}
void StencilOp(byte sfail,
               byte spass);
\end{verbatim}
\end{framed}

\begin{tabular}{lp{0.8 \textwidth}}

\\ \mach{sfail} & the operation to be applied to pixels that fail the stencil test.
One of \mach{ZERO}, \mach{KEEP}, \mach{REPLACE}, \mach{INCR}, \mach{DECR} or \mach{INVERT}.

\\ \mach{spass} & the operation to be applied to pixels that pass the stencil test.
One of \mach{ZERO}, \mach{KEEP}, \mach{REPLACE}, \mach{INCR}, \mach{DECR} or \mach{INVERT}.

\end{tabular}

\vspace{10pt}
The \mach{StencilOp} command controls how drawn pixels modify the stencil buffer.
If the the pixel failed the stencil test, then the operation specified by \mach{sfail} is performed.
Otherwise the operation \mach{spass} is performed.

\minipng{0038}
\eg{examples-stencil}


\drawcmd{TagMask}{Sets the mask controlling tag writes}

\begin{framed}
\begin{verbatim}
void TagMask(byte mask);
\end{verbatim}
\end{framed}

\begin{tabular}{lp{0.8 \textwidth}}

\\ \mach{mask} & if set, drawn pixels write to the tag buffer \\

\end{tabular}

\vspace{10pt}
The \mach{TagMask} command controls writes to the tag buffer.
If \mach{mask} is 1, then as each pixel is drawn the byte value set by \nameref{Tag} is also written to the tag buffer.
If the \mach{mask} is 0, then drawing does not affect the tag buffer.



\drawcmd{Tag}{Set the tag value for drawing}

\begin{framed}
\begin{verbatim}
void Tag(byte s);
\end{verbatim}
\end{framed}

\begin{tabular}{lp{0.8 \textwidth}}

\\ \mach{s} & tag value, 0-255 \\

\end{tabular}

\vspace{10pt}
The \mach{Tag} command sets the byte value that is drawn into the tag buffer.



\drawcmd{Vertex2f}{Draw at a subpixel position}

\begin{framed}
\begin{verbatim}
void Vertex2f(int16_t x,
              int16_t y);
\end{verbatim}
\end{framed}

\begin{tabular}{lp{0.8 \textwidth}}

\\ \mach{x} & $x$ coordinate of vertex in $1/16$ths of a pixel. -16384 to 16383. \\

\\ \mach{y} & $y$ coordinate of vertex in $1/16$ths of a pixel. -16384 to 16383. \\

\end{tabular}

\vspace{10pt}
The \mach{Vertex2f} command specifies a screen position for drawing.
What gets drawn depends on the current drawing object specified in \nameref{Begin}.
\mach{Vertex2f} specifies \textit{subpixel} coordinates, so it has a precision of $1/16$th of a pixel.
It also allows a coordinate range much larger than the physical screen - this is useful for drawing objects that are larger than the screen itself.

When drawing \mach{BITMAP} \mach{Vertex2f} uses the bitmap handle and cell currently set by
\nameref{BitmapHandle} and \nameref{Cell}.

\drawcmd{Vertex2ii}{Draw at a integer pixel position}

\begin{framed}
\begin{verbatim}
void Vertex2ii(uint16_t x,
               uint16_t y,
               byte handle = 0,
               byte cell = 0);
\end{verbatim}
\end{framed}

\begin{tabular}{lp{0.8 \textwidth}}

\\ \mach{x} & $x$ coordinate of vertex in pixels, 0-511 \\

\\ \mach{y} & $y$ coordinate of vertex in pixels, 0-511 \\

\\ \mach{handle} & bitmap handle, 0-31 \\

\\ \mach{cell} & bitmap cell, 0-127 \\

\end{tabular}

\vspace{10pt}
The \mach{Vertex2ii} command specifies a screen position for drawing.
What gets drawn depends on the current drawing object specified in \nameref{Begin}.
\mach{Vertex2ii} 

When drawing \mach{BITMAP} \mach{Vertex2ii} uses the bitmap handle and cell specified
in \mach{handle} and \mach{cell}. The graphics state of commands \nameref{BitmapHandle}
and \nameref{Cell} are neither used not altered by this command.
