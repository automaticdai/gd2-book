\drawcmd{AlphaFunc}{设置透明度测试功能}

\begin{framed}
\begin{verbatim}
void AlphaFunc(byte func,
               byte ref);
\end{verbatim}
\end{framed}

\begin{tabular}{lp{0.8 \textwidth}}

\\ \mach{func} & 检测采用的比较方式,可用参数有 \mach{NEVER}、 \mach{LESS}、 \mach{LEQUAL}、 \mach{GREATER}、 \mach{GEQUAL}、 \mach{EQUAL}、 \mach{NOTEQUAL} 或 \mach{ALWAYS}\\

\\ \mach{ref} & 比较时参考的数值 \\

\end{tabular}

\vspace{10pt}
透明度检测功能用于检测所有像素点的透明度值,并只绘制出满足条件的像素点。
例如:
\begin{verbatim}
  GD.AlphaFunc(GEQUAL, 160);
\end{verbatim}
表示只选择绘制透明度值满足 $(A \ge 160)$ 的像素点。
默认状态下的比较方式是 \mach{ALWAYS} ,即绘制所有像素点。
\minipng{0032}
\eg{examples-alphafunc}

\drawcmd{Begin}{选取用于绘制的绘图元素}

\begin{framed}
\begin{verbatim}
void Begin(byte prim);
\end{verbatim}
\end{framed}

\begin{tabular}{lp{0.8 \textwidth}}

\\ \mach{prim} &  
\mach{BITMAPS}、
\mach{POINTS}、
\mach{LINES}、
\mach{LINE\_STRIP}、
\mach{EDGE\_STRIP\_R}、
\mach{EDGE\_STRIP\_L}、
\mach{EDGE\_STRIP\_A}、
\mach{EDGE\_STRIP\_B} 或
\mach{RECTS} 之一\\
\end{tabular}

\vspace{10pt}
\mach{Begin} 指令用于设置当前绘制的图形种类。
这一指令并不绘制任何图像——绘制命令由之后会介绍的 \nameref{Vertex2f} 或 \nameref{Vertex2ii} 指令实现。
其中可供绘制的图元包括:

\vspace{10pt}

\begin{tabular}{lp{0.8 \textwidth}}
\mach{BITMAPS     } & 每次调用vertex指令时将绘制一幅位图图片\\
\mach{POINTS      } & 每次调用vertex指令时将绘制一个平滑边缘的点\\
\mach{LINES       } & 每次调用一对vertex指令时将绘制一条平滑线段\\
\mach{RECTS       } & 每次调用一对vertex指令时将绘制一个平滑边缘的矩形\\
\mach{LINE\_STRIP  } & 调用一组vertex指令时将按顺序绘制一条首尾相接的折线\\
\mach{EDGE\_STRIP\_A} & 同 \mach{LINE\_STRIP} 类似, 不过是对指定线上方的像素点进行绘制 \\
\mach{EDGE\_STRIP\_B} & 同 \mach{LINE\_STRIP} 类似, 不过是对指定线下方的像素点进行绘制 \\
\mach{EDGE\_STRIP\_L} & 同 \mach{LINE\_STRIP} 类似, 不过是对指定线左侧的像素点进行绘制 \\
\mach{EDGE\_STRIP\_R} & 同 \mach{LINE\_STRIP} 类似, 不过是对指定线右侧的像素点进行绘制
\end{tabular}

\png{0047}

\drawcmd{BitmapHandle}{设置位图句柄}

\begin{framed}
\begin{verbatim}
void BitmapHandle(byte handle);
\end{verbatim}
\end{framed}

\begin{tabular}{lp{0.8 \textwidth}}

\\ \mach{handle} & 整型句柄编号,范围为:$0\sim15$ \\

\end{tabular}

\vspace{10pt}
\mach{BitmapHandle} 指令用于设置位图句柄,位图句柄可用于
\nameref{Vertex2f} ,
\nameref{BitmapSource} ,
\nameref{BitmapLayout} 和
\nameref{BitmapSize} 。
位图句柄是图形环境的一部分,其默认值为 $0$ 。

\drawcmd{BitmapLayout}{设定位图的内存分布}

\begin{framed}
\begin{verbatim}
void BitmapLayout(byte format,
                  uint16_t linestride,
                  uint16_t height);
\end{verbatim}
\end{framed}

\begin{tabular}{lp{0.8 \textwidth}}

\\ \mach{format} & 位图像素的格式,
\mach{ARGB1555} 、
\mach{L1} 、
\mach{L4} 、
\mach{L8} 、
\mach{RGB332} 、
\mach{ARGB2} 、
\mach{ARGB4} 、
\mach{RGB565} 、
\mach{PALETTED} 、
\mach{TEXT8X8} 、
\mach{TEXTVGA} 、
\mach{BARGRAPH} 中选择其一
\\

\\ \mach{linestride} & 位图图像中每一行数据在存储器中的大小,单位为字节 \\

\\ \mach{height} & 位图的高度,单位为像素 \\

\end{tabular}

\vspace{10pt}
\mach{BitmapLayout} 指令用于设定位图在存储器中的分布。
\mach{format} 用于控制存储器数据如何转化为像素。
每一个图形像素的位数可以为 $1$、$4$、$8$ 或是 $16$ 位。
色彩信息可以从每个像素的数据中抽取得到,对应关系如下所示,
其中``v''表示每像素点的数据。

\vspace{10pt}
\begin{tabular}{cccccc}
格式           & 每像素图像 & alpha        & 红   & 绿 & 蓝 \\
                 &的位数     &              &       &       &      \\
\hline
\mach{L1}        &    1     & v$_0$        &  1.0  &  1.0  &  1.0 \\
\mach{L4}        &    4     & v$_{3..0}$   &  1.0  &  1.0  &  1.0 \\
\mach{L8}        &    8     & v$_{7..0}$   &  1.0  &  1.0  &  1.0 \\
\hline
\mach{RGB332}    &    8     & 1.0          &  v$_{7..5}$   &  v$_{4..2}$  &  v$_{1..0}$ \\
\mach{RGB565}    &    16    & 1.0          &  v$_{15..1}$  &  v$_{10..5}$ &  v$_{4..0}$ \\
\hline
\mach{ARGB2}     &    8     & v$_{7..6}$   &  v$_{5..4}$   &  v$_{3..2}$  &  v$_{1..0}$ \\
\mach{ARGB4}     &    16    & v$_{15..12}$ &  v$_{11..8}$  &  v$_{7..4}$  &  v$_{3..0}$ \\
\end{tabular}
\vspace{10pt}

$1$ 位和 $4$ 位的像素点按照从左至右的顺序打包成为字节,
因此字节中最左边的像素占据该字节的最高位。
$16$ 位图像在显存中以小端存储格式存储,而且存储时必须按照偶数内存边界对齐。

\mach{PALETTED} 格式每像素占 $8$ 位大小,每一像素数据都对应 \mach{RAM\_PALETTE}中存储的 $256$ 个 $32$ 位颜色之一。
% \mach{TEXT8X8} interpret each byte in the bitmap as an index into a character array, with a character size of 8x8 pixels.
% \mach{TEXTVGA} interprets memory in the same format as an IBM PC VGA graphic controller, 8 bits of character attribute and an 8-bit character code. The character size is 8x16.


\drawcmd{BitmapSize}{设置位图的尺寸及外观}

\begin{framed}
\begin{verbatim}
void BitmapSize(byte filter,
                byte wrapx,
                byte wrapy,
                uint16_t width,
                uint16_t height);
\end{verbatim}
\end{framed}

\begin{tabular}{lp{0.8 \textwidth}}

\\ \mach{filter} & 位图的过滤方式,\mach{NEAREST} 或 \mach{BILINEAR} \\

\\ \mach{wrapx} & $x$ 轴方向滚动模式, \mach{BORDER} 或 \mach{REPEAT} \\

\\ \mach{wrapy} & $x$ 轴方向滚动模式, \mach{BORDER} 或 \mach{REPEAT} \\

\\ \mach{width} & 屏幕上绘制的宽度(像素) \\

\\ \mach{height} & 屏幕上绘制的高度(像素) \\

\end{tabular}

\vspace{10pt}
\mach{BitmapSize} 指令用于设置当前位图在屏幕上的显示方式。


\drawcmd{BitmapSource}{设置位图的源地址}

\begin{framed}
\begin{verbatim}
void BitmapSource(uint32_t addr);
\end{verbatim}
\end{framed}

\begin{tabular}{lp{0.8 \textwidth}}

\\ \mach{addr} & 位图的基地址,取值范围为 \mach{0x00000} $\sim$ \mach{0x3ffff} \\

\end{tabular}

\vspace{10pt}
\mach{BitmapSource} 指令用于设置源位图的基地址。
对于 $16$ 位数据格式的位图文件,该地址必须为偶数。

\drawcmd{BlendFunc}{设定颜色混合功能}

\begin{framed}
\begin{verbatim}
void BlendFunc(byte src,
               byte dst);
\end{verbatim}
\end{framed}

\begin{tabular}{lp{0.8 \textwidth}}

\\ \mach{src} & 源混合因子,可选参数为
\mach{ZERO}、
\mach{ONE}、
\mach{SRC\_ALPHA}、
\mach{DST\_ALPHA}、
\mach{ONE\_MINUS\_SRC\_ALPHA} 和
\mach{ONE\_MINUS\_DST\_ALPHA}\\

\\ \mach{dst} & 目标混合因子,可选参数为
\mach{ZERO}、
\mach{ONE}、
\mach{SRC\_ALPHA}、
\mach{DST\_ALPHA}、
\mach{ONE\_MINUS\_SRC\_ALPHA} 及
\mach{ONE\_MINUS\_DST\_ALPHA}\\

\end{tabular}

\vspace{10pt}
\mach{BlendFunc} 指令用于将指定图像与当前帧缓冲器的内容进行图像混合。
输入图像的颜色与源混合因子相乘,而当前帧缓冲器中的图像与目标混合因子相乘,
最后再将这两部分的结果相加得到最终图像。

\minipng{0031}
\eg{examples-blendfunc}


% \drawcmd{Call}{Something}
% 
% \begin{framed}
% \begin{verbatim}
% void Call(uint16_t dest);
% \end{verbatim}
% \end{framed}
% 
% \begin{tabular}{lp{0.8 \textwidth}}
% 
% \\ \mach{dest} & something \\
% 
% \end{tabular}
% 
% \vspace{10pt}
% The \mach{Call} command

\drawcmd{Cell}{设置位图单元}

\begin{framed}
\begin{verbatim}
void Cell(byte cell);
\end{verbatim}
\end{framed}

\begin{tabular}{lp{0.8 \textwidth}}

\\ \mach{cell} & 位图单元号,取值范围为 $0\sim127$ \\

\end{tabular}

\vspace{10pt}
\mach{Cell} 指令用于设定当前 \nameref{Vertex2f} 指令使用的位图单元。

\png{0034}
\eg{examples-cell}


\drawcmd{ClearColorA}{设定清屏背景色的alpha通道分量}


\begin{framed}
\begin{verbatim}
void ClearColorA(byte alpha);
\end{verbatim}
\end{framed}

\begin{tabular}{lp{0.8 \textwidth}}

\\ \mach{alpha} & 清屏背景色的alpha通道分量,取值范围为 $0\sim255$ \\

\end{tabular}

\vspace{10pt}
\mach{ClearColorA} 指令用于设定清屏背景色的Alpha通道值。
随后调用的 \nameref{Clear} 函数会将此数值写入显示缓冲器的alpha通道中。


\drawcmd{Clear}{清屏}

\begin{framed}
\begin{verbatim}
void Clear(byte c = 1,
           byte s = 1,
           byte t = 1);
\end{verbatim}
\end{framed}

\begin{tabular}{lp{0.8 \textwidth}}

\\ \mach{c} & 设定该位时,清空颜色缓冲区 \\

\\ \mach{s} & 设定该位时,清空模板缓冲区 \\

\\ \mach{t} & 设定该位时,清空标签缓冲区 \\

\end{tabular}

\vspace{10pt}
\mach{Clear} 指令用于清空指定的帧缓冲器。

\png{0035}
\eg{examples-clear}


\drawcmd{ClearColorRGB}{设定清屏背景色的红、绿、蓝通道各分量}

\begin{framed}
\begin{verbatim}
void ClearColorRGB(byte red,
                   byte green,
                   byte blue);
void ClearColorRGB(uint32_t rgb);
\end{verbatim}
\end{framed}

\begin{tabular}{lp{0.8 \textwidth}}

\\ \mach{red} & 红色通道值,取值范围为 $0\sim255$ \\

\\ \mach{green} & 绿色通道值,取值范围为 $0\sim255$ \\

\\ \mach{blue} & 蓝色通道值,取值范围为 $0\sim255$ \\

\\ \mach{rgb} & 24位色,顺序依次为红、绿、蓝,\\
              & 取值范围为 \mach{0x000000} $\sim$ \mach{0xffffff} \\

\end{tabular}

\vspace{10pt}
\mach{ClearColorRGB} 指令用于设定清屏背景色的红、绿、蓝色彩通道分量值。
随后调用 \nameref{Clear} 函数将会把这些设定值写入帧缓冲器的色彩通道中。
\minipng{0030}
\eg{examples-clearcolorrgb}



\drawcmd{ClearStencil}{设定模板的清除值}

\begin{framed}
\begin{verbatim}
void ClearStencil(byte s);
\end{verbatim}
\end{framed}

\begin{tabular}{lp{0.8 \textwidth}}

\\ \mach{s} & 需要清除模板缓冲区中的模板值,取值范围为 $0\sim255$ \\

\end{tabular}

\vspace{10pt}
\mach{ClearStencil} 指令用于设定需要清除的模板值。
随后调用 \nameref{Clear} 函数将会把这一值写入当前模板缓冲器中。


\drawcmd{ClearTag}{设定需要清除的标签值}

\begin{framed}
\begin{verbatim}
void ClearTag(byte s);
\end{verbatim}
\end{framed}

\begin{tabular}{lp{0.8 \textwidth}}

\\ \mach{s} & 标签值,取值范围为 $0\sim255$ \\

\end{tabular}

\vspace{10pt}
\mach{ClearTag} 指令用于设定需要清除的标签值。
随后调用 \nameref{Clear} 将这一值写入当前标签缓冲器中。


\drawcmd{ColorA}{设定当前色彩设置的Alpha通道值}

\begin{framed}
\begin{verbatim}
void ColorA(byte alpha);
\end{verbatim}
\end{framed}

\begin{tabular}{lp{0.8 \textwidth}}

\\ \mach{alpha} & alpha通道值,取值范围为 $0\sim255$ \\

\end{tabular}

\vspace{10pt}
\mach{ColorA} 指令用于设定当前绘制色彩设置的Alpha通道值。
\png{0029}
\eg{examples-colora}

\drawcmd{ColorMask}{设定控制色彩通道的颜色蒙板}

\begin{framed}
\begin{verbatim}
void ColorMask(byte r,
               byte g,
               byte b,
               byte a);
\end{verbatim}
\end{framed}

\begin{tabular}{lp{0.8 \textwidth}}

\\ \mach{r} & 设定该位时,使能对红色色彩通道的写入 \\

\\ \mach{g} & 设定该位时,使能对绿色色彩通道的写入 \\

\\ \mach{b} & 设定该位时,使能对蓝色色彩通道的写入 \\

\\ \mach{a} & 设定该位时,使能对alpha通道的写入 \\

\end{tabular}

\vspace{10pt}
\mach{ColorMask} 指令用于设定色彩通道的颜色蒙板,使能 / 禁止对帧缓冲器中红、绿、蓝及alpha通道的写入。
\minipng{0033}
\eg{examples-colormask}


\drawcmd{ColorRGB}{设定当前色彩的红、绿、蓝通道值}

\begin{framed}
\begin{verbatim}
void ColorRGB(byte red,
              byte green,
              byte blue);

void ColorRGB(uint32_t rgb);
\end{verbatim}
\end{framed}

\begin{tabular}{lp{0.8 \textwidth}}

\\ \mach{red} & 红色通道值,取值范围为 $0\sim255$ \\

\\ \mach{green} & 绿色通道值,取值范围为 $0\sim255$ \\

\\ \mach{blue} & 蓝色通道值,取值范围为 $0\sim255$ \\

\\ \mach{rgb} & $24$ 位色,顺序依次为红、绿、蓝,\\
              & 取值范围为 \mach{0x000000} $\sim$ \mach{0xffffff} \\

\end{tabular}

\vspace{10pt}
\mach{ColorRGB} 指令用于设定当前色彩设置。
这一颜色设置适用于所有绘图操作。

\minipng{0028}
\eg{examples-colorrgb}


\drawcmd{LineWidth}{设定绘制线条的宽度}

\begin{framed}
\begin{verbatim}
void LineWidth(uint16_t width);
\end{verbatim}
\end{framed}

\begin{tabular}{lp{0.8 \textwidth}}

\\ \mach{width} & 线宽,单位为 $1/16$ 像素 \\

\end{tabular}

\vspace{10pt}
\mach{LineWidth} 指令用于设定绘制 \mach{LINES} 和 \mach{LINE\_STRIP} 时的线条宽度。
宽度单位为 $1/16$ 像素,即 \mach{LineWidth(16)} 表示 $1$ 像素的线宽。
值得注意的是,此线宽代表线条从其宽的中心到外边缘的距离,概念类似圆的半径。因此线条的实际宽度是指定值的两倍。

最大支持的线宽值为 $4095$ ,即 $255 \frac{15}{16}$ 像素。
\png{0026}
\eg{examples-linewidth}

\drawcmd{PointSize}{设定点的大小}

\begin{framed}
\begin{verbatim}
void PointSize(uint16_t size);
\end{verbatim}
\end{framed}

\begin{tabular}{lp{0.8 \textwidth}}

\\ \mach{size} & 点的大小,单位为 $1/16$ 像素 \\

\end{tabular}

\vspace{10pt}
\mach{PointSize} 指令用于设定绘制 \mach{POINTS} 时点的大小。
其大小单位为 $1/16$ 像素,即 \mach{PointSize(16)} 表示 $1$ 像素的点大小。
值得注意的是,此大小代表从点的中心到其外边缘的距离,即半径。因此点的直径是指定值的两倍。

最大值为 $4095$ 即 $255 \frac{15}{16}$ 像素。
\png{0027}
\eg{examples-pointsize}


\drawcmd{RestoreContext}{恢复绘制环境为先前保存的状态}

\begin{framed}
\begin{verbatim}
void RestoreContext(void);
\end{verbatim}
\end{framed}

\begin{tabular}{lp{0.8 \textwidth}}

\end{tabular}

\vspace{10pt}
图形显示的属性集合统称为图形环境( \textit{graphics context} )。
由 \mach{SaveContext} 将这一属性集合存储一份备份, 而 \mach{RestoreContext} 指令用于将图形环境恢复为这一备份的状态。

设备支持存储最多4份备份。其图形环境内容包括:
\vspace{10pt}

\begin{tabular}{rl}
属性 & 相应绘图指令 \\
\hline
\\
透明度测试功能   &  \nameref{AlphaFunc} \\
位图句柄         &  \nameref{BitmapHandle} \\
图像混合功能        &  \nameref{BlendFunc} \\
位图单元           &  \nameref{Cell} \\
清屏背景色彩值     & \nameref{ClearColorA}, \nameref{ClearColorRGB} \\
清除模板值   & \nameref{ClearStencil} \\
清除标签值       & \nameref{ClearTag} \\
色彩图层蒙版      & \nameref{ColorMask} \\
绘图颜色                 & \nameref{ColorA}, \nameref{ColorRGB} \\
线宽            & \nameref{LineWidth} \\
点大小            & \nameref{PointSize} \\
裁切矩形窗口     & \nameref{ScissorSize}, \nameref{ScissorXY} \\
模板检测功能 & \nameref{StencilFunc} \\
模板图层蒙板    & \nameref{StencilMask} \\
模板操作     & \nameref{StencilOp} \\
标签蒙板        & \nameref{TagMask} \\
标签值             & \nameref{Tag}
\end{tabular}



\drawcmd{SaveContext}{保存图形环境}

\begin{framed}
\begin{verbatim}
void SaveContext(void);
\end{verbatim}
\end{framed}

\begin{tabular}{lp{0.8 \textwidth}}

\end{tabular}

\vspace{10pt}
图形显示的属性集合统称为图形环境( \textit{graphics context} )。
\mach{SaveContext} 指令用于将这一属性集合存储一份备份, 而 \mach{RestoreContext} 将图形环境恢复为这一备份的状态。

\png{0036}
\eg{examples-context}

\drawcmd{ScissorSize}{设定裁切矩形窗口的尺寸}

\begin{framed}
\begin{verbatim}
void ScissorSize(uint16_t width,
                 uint16_t height);
\end{verbatim}
\end{framed}

\begin{tabular}{lp{0.8 \textwidth}}

\\ \mach{width} & 裁切矩形窗口的宽度,单位为1像素,取值范围为 $0\sim512$ \\

\\ \mach{height} & 裁切矩形窗口的高度,单位为1像素,取值范围为 $0\sim512$ \\

\end{tabular}

\vspace{10pt}
\mach{ScissorSize} 指令用于设定裁切矩形窗口的尺寸规格。
裁切矩形窗口用于将绘图指令限制在指定的一个矩形区域中。

\png{0037}
\eg{examples-scissor}


\drawcmd{ScissorXY}{设定裁切矩形窗口的左上角的坐标位置}

\begin{framed}
\begin{verbatim}
void ScissorXY(uint16_t x,
               uint16_t y);
\end{verbatim}
\end{framed}

\begin{tabular}{lp{0.8 \textwidth}}

\\ \mach{x} & 裁切矩形窗口的左上角的 $x$ 轴坐标,取值范围为 $0\sim511$ \\

\\ \mach{y} & 裁切矩形窗口的左上角的 $y$ 轴坐标,取值范围为 $0\sim511$ \\

\end{tabular}

\vspace{10pt}
\mach{ScissorXY} 指令用于设定裁切矩形窗口的左上角的坐标位置。
裁切矩形窗口用于将绘图指令限制在指定的一个矩形区域中。

\drawcmd{StencilFunc}{设定模板检测功能}

\begin{framed}
\begin{verbatim}
void StencilFunc(byte func,
                 byte ref,
                 byte mask);
\end{verbatim}
\end{framed}

\begin{tabular}{lp{0.8 \textwidth}}

\\ \mach{func} & 设定模板检测操作的比较方式,
        \mach{NEVER} , \mach{LESS} , \mach{LEQUAL} , \mach{GREATER} , \mach{GEQUAL} , \mach{EQUAL} , \mach{NOTEQUAL} 或 \mach{ALWAYS} 中选择其一\\

\\ \mach{ref} & 设定用于比较的模板参考值,取值范围为 $0\sim255$ \\

\\ \mach{mask} & an 8-bit mask that is anded with both \mach{ref} and the pixel's stencil value before comparison, 0-255 \\
\\ \mach{mask} & 检测前加入 \mach{ref} 和图像模板值的 $8$ 位蒙板,取值范围为 $0\sim255$ \\

\end{tabular}

\vspace{10pt}
\mach{StencilFunc} 指令用于控制模板检测操作。
绘制期间,模板检测应用于每一像素的图像,如果检测失败,该像素图像将不被绘制。
设定 \mach{func} 为 \mach{ALWAYS} 表示绘制所有像素点。

\drawcmd{StencilMask}{设定用于控制对模板进行编写的图层蒙板}

\begin{framed}
\begin{verbatim}
void StencilMask(byte mask);
\end{verbatim}
\end{framed}

\begin{tabular}{lp{0.8 \textwidth}}

\\ \mach{mask} & 每一位对应控制激活模板缓冲区的对应位的写入操作 \\

\end{tabular}

\vspace{10pt}
\mach{StencilMask} 指令用于控制向模板缓冲区写入数据。
由于模板缓冲区有 $8$ 位空间, \mach{mask} 中的每一位对应控制激活模板缓冲区的对应位的写入操作。
因此 \mach{mask} 为 \mach{0x00} 时将禁止模板写入,而 \mach{mask} 为 \mach{0xff} 时将激活模板写入。

\drawcmd{StencilOp}{设定模板更新操作}

\begin{framed}
\begin{verbatim}
void StencilOp(byte sfail,
               byte spass);
\end{verbatim}
\end{framed}

\begin{tabular}{lp{0.8 \textwidth}}

\\ \mach{sfail} & 当某像素点的模板测试失败时对其进行的操作。
\mach{ZERO} , \mach{KEEP} , \mach{REPLACE} , \mach{INCR} , \mach{DECR} 或 \mach{INVERT} 中选择其一

\\ \mach{spass} & 当某像素点的模板测试通过时对其进行的操作。
\mach{ZERO} , \mach{KEEP} , \mach{REPLACE} , \mach{INCR} , \mach{DECR} 或 \mach{INVERT} 中选择其一

\end{tabular}

\vspace{10pt}
\mach{StencilOp} 指令用于控制对像素点进行绘制操作时如何修改模板缓冲区。
当某像素点的模板测试失败时,将执行 \mach{sfail} 指定的操作。
通过测试时,将执行 \mach{spass} 指定的操作。

\minipng{0038}
\eg{examples-stencil}


\drawcmd{TagMask}{设定标签图层蒙板}

\begin{framed}
\begin{verbatim}
void TagMask(byte mask);
\end{verbatim}
\end{framed}

\begin{tabular}{lp{0.8 \textwidth}}

\\ \mach{mask} & 设定该位时, 绘制的图像将向标签缓存中写入数据 \\

\end{tabular}

\vspace{10pt}
\mach{TagMask} 指令用于控制向标签缓存中写入数据。
当 \mach{mask} 为 $1$ 时,每当有像素点有绘图操作, \nameref{Tag} 将向标签缓存写入一字节值作为其标签值。
当 \mach{mask} 为 $0$ 时,绘制操作对标签缓存不会产生影响。

\drawcmd{Tag}{设定绘制对象的标签值}

\begin{framed}
\begin{verbatim}
void Tag(byte s);
\end{verbatim}
\end{framed}

\begin{tabular}{lp{0.8 \textwidth}}

\\ \mach{s} & 标签值,取值范围为 $0\sim255$ \\

\end{tabular}

\vspace{10pt}
\mach{Tag} 指令用于指定一字节值为绘制对象的标签值并存入标签缓存中。


\drawcmd{Vertex2f}{在子坐标系坐标下绘制图像}

\begin{framed}
\begin{verbatim}
void Vertex2f(int16_t x,
              int16_t y);
\end{verbatim}
\end{framed}

\begin{tabular}{lp{0.8 \textwidth}}

\\ \mach{x} & 绘制图像位置的 $x$ 坐标值,精确度为 $1/16$ 像素,\\
            & 取值范围为 $-16384\sim16383$ \\

\\ \mach{y} & 绘制图像位置的 $y$ 坐标值,精确度为 $1/16$ 像素,\\
            & 取值范围为 $-16384\sim16383$ \\

\end{tabular}

\vspace{10pt}
\mach{Vertex2f} 指令用于指定绘制图像的屏幕位置。
而绘制的图像由当前 \nameref{Begin} 指定的图像对象决定。
\mach{Vertex2f} 指定子坐标系( \textit{subpixel} )坐标,所以其精确度是 $1/16$ 像素。
除此以外,其支持的坐标范围要大于实际屏幕的范围——这对于绘制尺寸大于屏幕的图像对象十分有帮助。

绘制 \mach{BITMAP} 时, \mach{Vertex2f} 将调用由当前 \nameref{BitmapHandle} 和 \nameref{Cell} 指定的位图句柄和位图单元。

\drawcmd{Vertex2ii}{在整型坐标值像素点处绘制图像}

\begin{framed}
\begin{verbatim}
void Vertex2ii(uint16_t x,
               uint16_t y,
               byte handle = 0,
               byte cell = 0);
\end{verbatim}
\end{framed}

\begin{tabular}{lp{0.8 \textwidth}}

\\ \mach{x} & 绘制图像位置的 $x$ 坐标值,取值范围为 $0\sim511$ \\

\\ \mach{y} & 绘制图像位置的 $y$ 坐标值,取值范围为 $0\sim511$ \\

\\ \mach{handle} & 位图句柄,取值范围为 $0\sim31$ \\

\\ \mach{cell} & 位图单元,取值范围为 $0\sim127$ \\

\end{tabular}

\vspace{10pt}
\mach{Vertex2ii} 指令用于指定绘制图像的屏幕位置。
而绘制的图像由当前 \nameref{Begin} 指定的图像对象决定。

绘制 \mach{BITMAP} 时, \mach{Vertex2ii} 将调用由 \mach{handle} 和 \mach{cell} 指定的位图句柄和位图单元。
而对于指令 \nameref{BitmapHandle} 和 \nameref{Cell} 中的图形属性信息,该指令既不会调用到也不会对其进行修改。

