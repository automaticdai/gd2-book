
\newcommand\aspect{
This widget draws itself horizontally if \mach{w} $>$ \mach{h}, otherwise it draws vertically.
}
\newcommand\threed{
By default this widget has a 3D look. Setting \mach{options} to \mach{OPT\_FLAT} gives it a 2D look.
}
\newcommand\fsixteen{
These arguments are expressed as signed 16.16 fixed
point values, so 65536 means 1.0.
As a convenience, the macro \mach{F16()} converts from floating-point to
signed 16.16 representation.
}
\newcommand\style{
All widgets share a common style;
they use the current background color (\nameref{cmd:bgcolor}) for non-interactive elements,
and the current foreground color (\nameref{cmd:fgcolor}) for interactive elements.
Using a darker color for \nameref{cmd:bgcolor} and a lighter one for \nameref{cmd:fgcolor} is probably a good idea.
}
\newcommand\centeropts{
If \mach{options} is \mach{OPT\_CENTERX} the text is centered horizontally,
if \mach{OPT\_CENTERY} then vertically, and if \mach{OPT\_CENTER} then both
horizontally and vertically.
Adding \mach{OPT\_SIGNED} causes \mach{val} to be treated as signed,
and displayed with a leading minus sign if negative.
}
\newcommand{\sbs}[2]{
\begin{minipage}[][][t]{0.70\textwidth}
#2
\end{minipage}
\begin{minipage}[][][t]{0.30\textwidth}
\begin{center}
\includegraphics[width=0.9\textwidth]{assets/#1.png}
\end{center}
\end{minipage}
}
\newcommand\example[2]{
For an example of \mach{cmd\_#1()}, see \nameref{#2}.
}

\highcmd{append}{Something}

\begin{framed}
\begin{verbatim}
void cmd_append(uint32_t ptr,
                uint32_t num);
\end{verbatim}
\end{framed}

The \mach{append} command
executes \mach{num} bytes of drawing commands from graphics memory at
\mach{ptr}.
This can be useful for using graphics memory as a cache for frequently
used drawing sequences, much like OpenGL's display lists.



\highcmd{bgcolor}{Something}

\begin{framed}
\begin{verbatim}
void cmd_bgcolor(uint32_t c);
\end{verbatim}
\end{framed}

The \mach{bgcolor} command sets the background color used when drawing widgets.
\style


\highcmd{button}{Something}

\begin{framed}
\begin{verbatim}
void cmd_button(int16_t x,
                int16_t y,
                uint16_t w,
                uint16_t h,
                byte font,
                uint16_t options,
                const char *label);
\end{verbatim}
\end{framed}

\sbs{0024}{
The \mach{button} command
draws a button widget at screen (\mach{x}, \mach{y}) with pixel size \mach{w} $\times$ \mach{h}.
\mach{label} gives the text label.
\threed
}

\highcmd{calibrate}{Something}

\begin{framed}
\begin{verbatim}
void cmd_calibrate(void);
\end{verbatim}
\end{framed}

The \mach{calibrate} command
runs the GPU's interactive touchscreen calibration procedure.



\highcmd{clock}{Something}

\begin{framed}
\begin{verbatim}
void cmd_clock(int16_t x,
               int16_t y,
               int16_t r,
               uint16_t options,
               uint16_t h,        // hours 0-23
               uint16_t m,        // minutes 0-59
               uint16_t s,        // seconds 0-59
               uint16_t ms);      // milliseconds 0-999
\end{verbatim}
\end{framed}

\sbs{0013}{
The \mach{clock} command
draws an analog clock at screen (\mach{x}, \mach{y}) with pixel radius \mach{r}.
The displayed time is \mach{h}, \mach{m}, \mach{s} and \mach{ms}.
\threed
}

\highcmd{coldstart}{Something}

\begin{framed}
\begin{verbatim}
void cmd_coldstart(void);
\end{verbatim}
\end{framed}

The \mach{coldstart} command
resets all widget state to its default value.


\highcmd{dial}{Something}

\begin{framed}
\begin{verbatim}
void cmd_dial(int16_t x,
              int16_t y,
              int16_t r,
              uint16_t options,
              uint16_t val);
\end{verbatim}
\end{framed}

\sbs{0014}{
The \mach{dial} command
draws a dial at screen (\mach{x}, \mach{y}) with pixel radius \mach{r}.
\mach{val} gives the dial's angular position in Furmans.
\threed
}

\highcmd{fgcolor}{Something}

\begin{framed}
\begin{verbatim}
void cmd_fgcolor(uint32_t c);
\end{verbatim}
\end{framed}

The \mach{fgcolor} command
Sets the foreground color used for drawing widgets.
\style

\highcmd{gauge}{Something}

\begin{framed}
\begin{verbatim}
void cmd_gauge(int16_t x,
               int16_t y,
               int16_t r,
               uint16_t options,
               uint16_t major,
               uint16_t minor,
               uint16_t val,
               uint16_t range);
\end{verbatim}
\end{framed}

\sbs{0015}{
The \mach{gauge} command
draws an analog gauge at screen (\mach{x}, \mach{y}) with pixel radius \mach{r}.
\mach{major} and \mach{minor}
are the number of
major and minor tick marks on the gauge's face.
The fraction (\mach{val} / \mach{range}) gives the gauge's value.
\threed
}


\highcmd{getprops}{Something}

\begin{framed}
\begin{verbatim}
void cmd_getprops(uint32_t &ptr,
                  uint32_t &w,
                  uint32_t &h);
\end{verbatim}
\end{framed}

The \mach{getprops} command
queries the GPU for the properties of the last image loaded by
\nameref{cmd:loadimage}.
\mach{ptr} is the image base address, and (\mach{w}, \mach{h}) gives its size in pixels.


\highcmd{gradient}{Something}

\begin{framed}
\begin{verbatim}
void cmd_gradient(int16_t x0,
                  int16_t y0,
                  uint32_t rgb0,
                  int16_t x1,
                  int16_t y1,
                  uint32_t rgb1);
\end{verbatim}
\end{framed}

\sbs{0012}{
The \mach{gradient} command
draws a smooth color gradient, blended from
color \mach{rgb0} at screen pixel (\mach{x0}, \mach{y0})
to \mach{rgb1} at (\mach{x1}, \mach{y1}).
\example{gradient}{gradient}
}


\highcmd{inflate}{Something}

\begin{framed}
\begin{verbatim}
void cmd_inflate(uint32_t ptr);
\end{verbatim}
\end{framed}

The \mach{inflate} command
decompresses data into main graphics memory at \mach{ptr}.
The compressed data should be supplied
after this command.
The format of the compressed data is zlib DEFLATE.

\highcmd{keys}{Something}

\begin{framed}
\begin{verbatim}
void cmd_keys(int16_t x,
              int16_t y,
              int16_t w,
              int16_t h,
              byte font,
              uint16_t options,
              const char *keys);
\end{verbatim}
\end{framed}

\sbs{0016}{
The \mach{keys} command
draws a rows of keys, each labeled with the characters of \mach{chars},
at screen (\mach{x}, \mach{y}) with pixel size \mach{w} $\times$ \mach{h}.
\threed
Specifying a character code in \mach{options} highlights that key.
}

\highcmd{loadidentity}{Something}

\begin{framed}
\begin{verbatim}
void cmd_loadidentity(void);
\end{verbatim}
\end{framed}

The \mach{loadidentity} command
sets the bitmap transform to the identity matrix.


\highcmd{loadimage}{Something}
\index{jpeg}

\begin{framed}
\begin{verbatim}
void cmd_loadimage(uint32_t ptr,
                   int32_t options);
\end{verbatim}
\end{framed}

\sbs{jpeg2}{
The \mach{loadimage} command
uncompresses a JPEG image into graphics memory at address \mach{ptr}.
The image parameters are loaded into the current bitmap handle.
The default format for the image is \mach{RGB565}.
If \mach{options} is \mach{OPT\_MONO} then the format is \mach{L8}.
\example{loadimage}{bitmaphandles}
}

\highcmd{memcpy}{Something}

\begin{framed}
\begin{verbatim}
void cmd_memcpy(uint32_t dest,
                uint32_t src,
                uint32_t num);
\end{verbatim}
\end{framed}

The \mach{memcpy} command
copies \mach{num} bytes from \mach{src} to \mach{dest}
in graphics memory.


\highcmd{memset}{Something}

\begin{framed}
\begin{verbatim}
void cmd_memset(uint32_t ptr,
                byte value,
                uint32_t num);
\end{verbatim}
\end{framed}

The \mach{memset} command
sets \mach{num} bytes starting at \mach{ptr} to \mach{value}
in graphics memory.


\highcmd{memwrite}{Something}

\begin{framed}
\begin{verbatim}
void cmd_memwrite(uint32_t ptr,
                  uint32_t num);
\end{verbatim}
\end{framed}

The \mach{memwrite} command
copies the following \mach{num} bytes into
graphics memory, starting at addresss \mach{ptr}.


\highcmd{regwrite}{Something}

\begin{framed}
\begin{verbatim}
void cmd_regwrite(uint32_t ptr,
                  uint32_t val);
\end{verbatim}
\end{framed}

The \mach{regwrite} command
sets the GPU register \mach{ptr} to \mach{val}.



\highcmd{number}{Something}

\begin{framed}
\begin{verbatim}
void cmd_number(int16_t x,
                int16_t y,
                byte font,
                uint16_t options,
                uint32_t val);
\end{verbatim}
\end{framed}

\sbs{0017}{
The \mach{number} command
renders a decimal number \mach{val} in font \mach{font}
at screen (\mach{x}, \mach{y}).
If options is $n$, then leading zeroes are added so that $n$ digits 
are always drawn.
\centeropts
}

\highcmd{progress}{Something}

\begin{framed}
\begin{verbatim}
void cmd_progress(int16_t x,
                  int16_t y,
                  int16_t w,
                  int16_t h,
                  uint16_t options,
                  uint16_t val,
                  uint16_t range);
\end{verbatim}
\end{framed}

\sbs{0018}{
The \mach{progress} command
draws a progress bar 
at screen (\mach{x}, \mach{y}) with pixel size \mach{w} $\times$ \mach{h}.
The fraction (\mach{val} / \mach{range}) gives the bar's value.
\aspect
}

\highcmd{rotate}{Something}

\begin{framed}
\begin{verbatim}
void cmd_rotate(int32_t a);
\end{verbatim}
\end{framed}

The \mach{rotate} command
applies a rotation of \mach{a} Furmans to the
current bitmap transform matrix.

\highcmd{scale}{Something}

\begin{framed}
\begin{verbatim}
void cmd_scale(int32_t sx,
               int32_t sy);
\end{verbatim}
\end{framed}

The \mach{scale} command
scales the current bitmap transform matrix by
(\mach{sx}, \mach{sy}).
\fsixteen

\highcmd{scrollbar}{Something}

\begin{framed}
\begin{verbatim}
void cmd_scrollbar(int16_t x,
                   int16_t y,
                   int16_t w,
                   int16_t h,
                   uint16_t options,
                   uint16_t val,
                   uint16_t size,
                   uint16_t range);
\end{verbatim}
\end{framed}

\sbs{0019}{
The \mach{scrollbar} command
draws a scroll bar 
at screen (\mach{x}, \mach{y}) with pixel size \mach{w} $\times$ \mach{h}.
The fraction (\mach{val} / \mach{range}) gives the bar's value,
and (\mach{size} / \mach{range}) gives its size.
\aspect
}


\highcmd{setfont}{Something}

\begin{framed}
\begin{verbatim}
void cmd_setfont(byte font,
                 uint32_t ptr);
\end{verbatim}
\end{framed}

\vspace{20pt}
The \mach{setfont} command
defines a RAM \mach{font} numbered 0-15
using a font block at \mach{ptr}.
Before calling \mach{setfont}, the font graphics must be loaded into memory
and the bitmap handle must be set up.
The font block is 148 bytes in size, as follows:

\vspace{10 pt}
\begin{tabular}{lcl}
address & size & value \\ \hline
$\mach{ptr}+0$   & 128 & width of each font character, in pixels \\
$\mach{ptr}+128$ & 4   & font bitmap format, for example \mach{L1}, \mach{L4} or \mach{L8}  \\
$\mach{ptr}+132$ & 4   & font line stride, in bytes \\
$\mach{ptr}+136$ & 4   & font width, in pixels \\
$\mach{ptr}+140$ & 4   & font height, in pixels \\
$\mach{ptr}+144$ & 4   & pointer to font graphic data in memory \\
\end{tabular}


\highcmd{setmatrix}{Something}

\begin{framed}
\begin{verbatim}
void cmd_setmatrix(void);
\end{verbatim}
\end{framed}

\begin{tabular}{cl}

\end{tabular}

\vspace{20pt}
The \mach{setmatrix} command
applies the current bitmap transform matrix to the next drawing operation.


\highcmd{sketch}{Something}

\begin{framed}
\begin{verbatim}
void cmd_sketch(int16_t x,
                int16_t y,
                uint16_t w,
                uint16_t h,
                uint32_t ptr,
                uint16_t format);
\end{verbatim}
\end{framed}

\vspace{20pt}
\sbs{sketching_b}{
The \mach{sketch} command
starts a continuous sketching process
that paints touched pixels
into a bitmap in graphics memory.
The bitmap's base addresss is given in
\mach{ptr}, its size in
(\mach{w},  \mach{h})
and its position on screen is
(\mach{x}, \mach{y}).
The \mach{format} of the bitmap can be either
\mach{L1} or \mach{L8}.
}
The sketching continues until \nameref{cmd:stop} is executed.
For an example of \nameref{cmd:sketch}, see \xref{sketching}.

\highcmd{slider}{Something}

\begin{framed}
\begin{verbatim}
void cmd_slider(int16_t x,
                int16_t y,
                uint16_t w,
                uint16_t h,
                uint16_t options,
                uint16_t val,
                uint16_t range);
\end{verbatim}
\end{framed}

\sbs{0020}{
The \mach{slider} command
draws a control slider
at screen (\mach{x}, \mach{y}) with pixel size \mach{w} $\times$ \mach{h}.
The fraction (\mach{val} / \mach{range}) gives the slider's value.
\aspect
\threed
}



\highcmd{spinner}{Something}

\begin{framed}
\begin{verbatim}
void cmd_spinner(int16_t x,
                 int16_t y,
                 byte style,
                 byte scale);
\end{verbatim}
\end{framed}

\vspace{20pt}
\sbs{0021}{
The \mach{spinner} command
starts drawing an animated ``waiting'' spinner
centered at screen pixel (\mach{x}, \mach{y}).
\mach{style} gives the spinner style;
\mach{0} is circular,
\mach{1} is linear,
\mach{2} is a clock, and
\mach{3} is rotating disks.
\mach{scale} gives the size of the graphic;
\mach{0} is small and
\mach{2} is huge.

% Use \nameref{cmd:stop} to stop the spin animation.
}

\highcmd{stop}{Something}

\begin{framed}
\begin{verbatim}
void cmd_stop(void);
\end{verbatim}
\end{framed}

The \mach{stop} command stops the current animating spinner,
or the currently executing sketching operation.
See \nameref{cmd:spinner} and
\nameref{cmd:sketch}.



\highcmd{text}{Something}

\begin{framed}
\begin{verbatim}
void cmd_text(int16_t x,
              int16_t y,
              byte font,
              uint16_t options,
              const char s);
\end{verbatim}
\end{framed}

\sbs{helloworld}{
The \mach{text} command
draws a text string number \mach{s} in font \mach{font}
at screen (\mach{x}, \mach{y}).
\centeropts
}
\example{text}{helloworld}


\highcmd{toggle}{Something}

\begin{framed}
\begin{verbatim}
void cmd_toggle(int16_t x,
                int16_t y,
                int16_t w,
                byte font,
                uint16_t options,
                uint16_t state,
                const char *s);
\end{verbatim}
\end{framed}

\sbs{0023}{
The \mach{toggle} command
draws a toggle control
at screen (\mach{x}, \mach{y})
with width \mach{w} pixels.
The position of the toggle is given by \mach{state};
0 means the toggle is in the left position, 65535 in the right.
The label contains a pair of strings, separated by ASCII character code
\mach{0xff}.
}
\threed

\highcmd{track}{Something}

\begin{framed}
\begin{verbatim}
void cmd_track(int16_t x,
               int16_t y,
               uint16_t w,
               uint16_t h,
               byte tag);
\end{verbatim}
\end{framed}

The \mach{track} command
instructs the GPU to track presses
that touch a pixel with \mach{tag}
and report them in \mach{GD.inputs.track\_val}.
The screen rectangle at
(\mach{x}, \mach{y}) with size
(\mach{w}, \mach{h}) defines the track area.
If the track area is $1 \times 1$ pixels in size, then the tracking is angular,
and \mach{GD.inputs.track\_val} reports an angle in Furmans relative to the tracking center
(\mach{x}, \mach{y}).
Angular tracking is useful for the
\nameref{cmd:dial} and \nameref{cmd:clock} widgets, and for games with a rotating control element, e.g. \nameref{nightstrike}.

Otherwise, the tracking is linear along the long axis of the (\mach{w}, \mach{h}) rectangle, and the value reported in 
\mach{GD.inputs.track\_val} is the distance along the rectangle, normalized to the range 0-65535.
Linear tracking is useful for the
\nameref{cmd:scrollbar},
\nameref{cmd:toggle}
and
\nameref{cmd:slider} widgets.

For an example of \nameref{cmd:track}, see \nameref{widgets}.

\highcmd{translate}{Something}

\begin{framed}
\begin{verbatim}
void cmd_translate(int32_t tx,
                   int32_t ty);
\end{verbatim}
\end{framed}

The \mach{translate} command
applies a translation of (\mach{tx}, \mach{ty})
to the bitmap transform matrix.
\fsixteen


