
\newcommand\aspect{
当 $w>h$ 时,这一控件将自动水平绘制,否则将竖直绘制。
}
\newcommand\threed{
默认状态下,这一控件使用3D效果的外观显示。将 \mach{options} 设置为 \mach{OPT\_FLAT} 可令其显示2D效果。
}
\newcommand\fsixteen{
上述参数的格式为 $16.16$ 定点浮点数,即 $65536$ 表示 $1.0$ 。
方便起见,调用宏函数 \mach{F16()} 可将浮点型数据转换为有符号型 $16.16$ 的表达形式。
}
\newcommand\style{
所有控件均包含一些相同的样式;对于非交互式部件使用背景色( \nameref{cmd:bgcolor} )绘制,
对于交互式部件使用前景色( \nameref{cmd:fgcolor} )绘制。
通常情况下最好使用较深的颜色作为背景色( \nameref{cmd:bgcolor} ),较浅的颜色作为前景色( \nameref{cmd:fgcolor} )。
}
\newcommand\centeropts{
当 \mach{options} 设置为 \mach{OPT\_CENTERX} 时,文本将水平居中显示,
为 \mach{OPT\_CENTERY} 时,文本将垂直居中显示,
为 \mach{OPT\_CENTER} 时,文本既水平居中,又垂直居中显示。
添加 \mach{OPT\_SIGNED} 设置时, \mach{val} 的值将被识别为有符号型数据,
即其值为负数时,在数值前将显示一个负号。 
}
\newcommand{\sbs}[2]{
\begin{minipage}[][][t]{0.70\textwidth}
#2
\end{minipage}
\begin{minipage}[][][t]{0.30\textwidth}
\begin{center}
\includegraphics[width=0.9\textwidth]{assets/#1.png}
\end{center}
\end{minipage}
}
\newcommand\example[2]{
关于 \mach{cmd\_#1()} 的具体例程请参见 \nameref{#2} 的内容。
}

\highcmd{append}{Something}

\begin{framed}
\begin{verbatim}
void cmd_append(uint32_t ptr,
                uint32_t num);
\end{verbatim}
\end{framed}

\mach{append} 指令用于调用显存地址 \mach{ptr} 开始,长度为 \mach{num} 字节的绘图指令并执行。
这一方式相当于把显存作为高速缓存来使用,对于频繁调用同一段绘图指令的操作十分适用,很像OpenGL中的显示列表。


\highcmd{bgcolor}{Something}

\begin{framed}
\begin{verbatim}
void cmd_bgcolor(uint32_t c);
\end{verbatim}
\end{framed}

\mach{bgcolor} 指令用于设定绘制控件时使用的背景色。
\style

\highcmd{button}{Something}

\begin{framed}
\begin{verbatim}
void cmd_button(int16_t x,
                int16_t y,
                uint16_t w,
                uint16_t h,
                byte font,
                uint16_t options,
                const char *label);
\end{verbatim}
\end{framed}

\sbs{0024}{
\mach{button} 指令用于在屏幕坐标 $(x, y)$ 处绘制大小为 $w \times h$ 的按钮。
\mach{label} 用于设置文本的标签。
\threed
}

\highcmd{calibrate}{Something}

\begin{framed}
\begin{verbatim}
void cmd_calibrate(void);
\end{verbatim}
\end{framed}

\mach{calibrate} 指令用于运行GPU内置的交互式触摸屏校准程序。


\highcmd{clock}{Something}

\begin{framed}
\begin{verbatim}
void cmd_clock(int16_t x,
               int16_t y,
               int16_t r,
               uint16_t options,
               uint16_t h,        // hours 0-23
               uint16_t m,        // minutes 0-59
               uint16_t s,        // seconds 0-59
               uint16_t ms);      // milliseconds 0-999
\end{verbatim}
\end{framed}

\sbs{0013}{
\mach{clock} 指令用于在屏幕像素坐标 $(x, y)$ 处绘制一个半径为 \mach{r} 的模拟时钟。
显示的时间由\mach{h} 、 \mach{m} 、 \mach{s} 和 \mach{ms} 指定。
\threed
}

\highcmd{coldstart}{Something}

\begin{framed}
\begin{verbatim}
void cmd_coldstart(void);
\end{verbatim}
\end{framed}

\mach{coldstart} 指令用于将所有控件的状态重置为默认值。

\highcmd{dial}{Something}

\begin{framed}
\begin{verbatim}
void cmd_dial(int16_t x,
              int16_t y,
              int16_t r,
              uint16_t options,
              uint16_t val);
\end{verbatim}
\end{framed}

\sbs{0014}{
\mach{dial} 指令用于在屏幕像素坐标 $(x, y)$ 处绘制一个半径为 \mach{r} 的标度盘。
\mach{val} 用于设定标度盘上显示的角度位置,单位为Furmans角度。
\threed
}

\highcmd{fgcolor}{Something}

\begin{framed}
\begin{verbatim}
void cmd_fgcolor(uint32_t c);
\end{verbatim}
\end{framed}

\mach{fgcolor} 指令用于设定用于绘制控件时的前景色。
\style

\highcmd{gauge}{Something}

\begin{framed}
\begin{verbatim}
void cmd_gauge(int16_t x,
               int16_t y,
               int16_t r,
               uint16_t options,
               uint16_t major,
               uint16_t minor,
               uint16_t val,
               uint16_t range);
\end{verbatim}
\end{framed}

\sbs{0015}{
\mach{gauge} 指令用于绘制一个模拟仪表盘,
其位置位于屏幕像素坐标 $(x, y)$ 处,半径为 \mach{r} ,单位为像素。
\mach{major} 和 \mach{minor} 用于设定仪表盘上最大和最小刻度值。
分数值( \mach{val} / \mach{range} ) 为仪表盘指针显示值。
\threed
}


\highcmd{getprops}{Something}

\begin{framed}
\begin{verbatim}
void cmd_getprops(uint32_t &ptr,
                  uint32_t &w,
                  uint32_t &h);
\end{verbatim}
\end{framed}

\mach{getprops} 指令用于询问GPU上一张由 \nameref{cmd:loadimage} 指令加载的图片属性。
\mach{ptr} 为这一图片的基地址, 而 $(w, h)$ 为这一图片的尺寸大小,单位为像素。
\\
\\

\highcmd{gradient}{Something}

\begin{framed}
\begin{verbatim}
void cmd_gradient(int16_t x0,
                  int16_t y0,
                  uint32_t rgb0,
                  int16_t x1,
                  int16_t y1,
                  uint32_t rgb1);
\end{verbatim}
\end{framed}

\sbs{0012}{
\mach{gradient} 指令用于绘制一幅平滑变化的渐变色背景,由屏幕像素坐标 $(x0, y0)$ 处的颜色 \mach{rgb0} 
逐渐变化,直至屏幕像素坐标 $(x1, y1)$ 处渐变为颜色 \mach{rgb1}。
\example{gradient}{gradient}
}


\highcmd{inflate}{Something}

\begin{framed}
\begin{verbatim}
void cmd_inflate(uint32_t ptr);
\end{verbatim}
\end{framed}

\mach{inflate} 指令用于对压缩数据进行解压并存储至主存储器地址 \mach{ptr} 中。
调用该指令后应提供相应的压缩数据,压缩格式使用 zlib DEFLATE 。


\highcmd{keys}{Something}

\begin{framed}
\begin{verbatim}
void cmd_keys(int16_t x,
              int16_t y,
              int16_t w,
              int16_t h,
              byte font,
              uint16_t options,
              const char *keys);
\end{verbatim}
\end{framed}

\sbs{0016}{
\mach{keys} 指令用于绘制一排按键,每个按键依次由字符串(\mach{chars})的每个字母进行标注,
其位置位于屏幕像素坐标 $(x, y)$ 处,大小为$w \times h$。
\threed
在 \mach{options} 设置中制定一个字母代码将在按键中对其进行高亮显示。
}

\highcmd{loadidentity}{Something}

\begin{framed}
\begin{verbatim}
void cmd_loadidentity(void);
\end{verbatim}
\end{framed}

\mach{loadidentity} 指令用于将位图数据转换为单位矩阵数据。
\newpage

\highcmd{loadimage}{Something}
\index{jpeg}

\begin{framed}
\begin{verbatim}
void cmd_loadimage(uint32_t ptr,
                   int32_t options);
\end{verbatim}
\end{framed}

\sbs{jpeg2}{
\mach{loadimage} 指令用于将一幅JPEG图片解压到从显存地址 \mach{ptr} 开始的存储空间,
并将图片的相关参数存储至当前位图句柄。
图片的默认格式为 \mach{RGB565},
而当 \mach{options} 设置为 \mach{OPT\_MONO} 时,格式则是 \mach{L8}。
\example{loadimage}{bitmaphandles}
}

\highcmd{memcpy}{Something}

\begin{framed}
\begin{verbatim}
void cmd_memcpy(uint32_t dest,
                uint32_t src,
                uint32_t num);
\end{verbatim}
\end{framed}

\mach{memcpy} 指令用于将显存首地址为 \mach{src},长度为 \mach{num} 字节的数据复制到 \mach{dest} 开始的存储空间。

\highcmd{memset}{Something}

\begin{framed}
\begin{verbatim}
void cmd_memset(uint32_t ptr,
                byte value,
                uint32_t num);
\end{verbatim}
\end{framed}

\mach{memset} 指令用于将从显存地址 \mach{ptr} 开始长度为 \mach{num} 字节的数据值统一设置为 \mach{value} 。

\highcmd{memwrite}{Something}

\begin{framed}
\begin{verbatim}
void cmd_memwrite(uint32_t ptr,
                  uint32_t num);
\end{verbatim}
\end{framed}

\mach{memwrite} 指令用于将紧随其后长度为 \mach{num} 字节的数据,
拷贝到从显存地址 \mach{ptr} 开始的存储空间。

\highcmd{regwrite}{Something}

\begin{framed}
\begin{verbatim}
void cmd_regwrite(uint32_t ptr,
                  uint32_t val);
\end{verbatim}
\end{framed}

\mach{regwrite} 指令用于将GPU寄存器 \mach{ptr} 的值设定为 \mach{val} 。


\highcmd{number}{Something}

\begin{framed}
\begin{verbatim}
void cmd_number(int16_t x,
                int16_t y,
                byte font,
                uint16_t options,
                uint32_t val);
\end{verbatim}
\end{framed}

\sbs{0017}{
\mach{number} 指令用于显示一串数字 \mach{val} ,其字体由 \mach{font} 设定,
其位置位于屏幕像素坐标 $(x, y)$ 处。
当 \mach{options} 设置为 $n$ 时,将始终保持显示 $n$ 位数字,如果位数不够,数字前加 $0$ 补齐。
\centeropts
}

\highcmd{progress}{Something}

\begin{framed}
\begin{verbatim}
void cmd_progress(int16_t x,
                  int16_t y,
                  int16_t w,
                  int16_t h,
                  uint16_t options,
                  uint16_t val,
                  uint16_t range);
\end{verbatim}
\end{framed}

\sbs{0018}{
\mach{progress} 指令用于绘制进度条,其位置位于屏幕像素坐标 $(x, y)$ 处,
尺寸为 $w \times h$ ,单位为像素。
分数值( \mach{val} / \mach{range} )表示当前进度值。
\aspect
}

\highcmd{rotate}{Something}

\begin{framed}
\begin{verbatim}
void cmd_rotate(int32_t a);
\end{verbatim}
\end{framed}

\mach{rotate} 指令用于对当前位图变换矩阵应用旋转 \mach{a} Furmans角度的操作。

\highcmd{scale}{Something}

\begin{framed}
\begin{verbatim}
void cmd_scale(int32_t sx,
               int32_t sy);
\end{verbatim}
\end{framed}

\mach{scale} 指令用于以 $(sx, sy)$ 比例对当前位图变换矩阵进行缩放操作。
\fsixteen

\highcmd{scrollbar}{Something}

\begin{framed}
\begin{verbatim}
void cmd_scrollbar(int16_t x,
                   int16_t y,
                   int16_t w,
                   int16_t h,
                   uint16_t options,
                   uint16_t val,
                   uint16_t size,
                   uint16_t range);
\end{verbatim}
\end{framed}

\sbs{0019}{
\mach{scrollbar} 指令用于在屏幕像素坐标 $(x, y)$ 处绘制大小为 $w \times h$ 像素的滚动条。
分数值( \mach{val} / \mach{range} )用于设定其滚动滑块所处的位置,
而分数值( \mach{size} / \mach{range} )用于设定滚动滑块的尺寸。
\aspect
}


\highcmd{setfont}{Something}

\begin{framed}
\begin{verbatim}
void cmd_setfont(byte font,
                 uint32_t ptr);
\end{verbatim}
\end{framed}

\vspace{20pt}

\mach{setfont} 指令用于定义存储在内存中的标号 \mach{font} 介于 $0\sim15$ 的字体。该字体的属性由存储在内存 \mach{ptr} 处的字体块定义。
在调用 \mach{setfont} 前,字体库必须已经存储在内存中,且位图句柄必须已经设置好。
字体块的大小为 $148$ 字节,具体描述如下所示:

\vspace{10 pt}
\begin{tabular}{lcl}
地址 & 大小 & 数值 \\ \hline
\\
$\mach{ptr}+0$   & 128 & 每个字体字符的宽度,单位为像素 \\
$\mach{ptr}+128$ & 4   & 字体位图格式,例如 \mach{L1} 、 \mach{L4} 或 \mach{L8}  \\
$\mach{ptr}+132$ & 4   & 字体行间距,单位为字节 \\
$\mach{ptr}+136$ & 4   & 字体宽度,单位为像素 \\
$\mach{ptr}+140$ & 4   & 字体高度,单位为像素 \\
$\mach{ptr}+144$ & 4   & 指向内存中字体图形所在位置的指针 \\
\end{tabular}


\highcmd{setmatrix}{Something}

\begin{framed}
\begin{verbatim}
void cmd_setmatrix(void);
\end{verbatim}
\end{framed}

\begin{tabular}{cl}

\end{tabular}

\vspace{20pt}
\mach{setmatrix} 指令用于将当前位图变换矩阵应用到下次绘图操作中。

\highcmd{sketch}{Something}

\begin{framed}
\begin{verbatim}
void cmd_sketch(int16_t x,
                int16_t y,
                uint16_t w,
                uint16_t h,
                uint32_t ptr,
                uint16_t format);
\end{verbatim}
\end{framed}

\vspace{20pt}
\sbs{sketching_b}{
\mach{sketch} 指令用于开始一次连续的绘图操作。该指令会将之后所有触摸到的像素点记录、绘制到图形内存中。
位图的基址由 \mach{ptr} 给定,
其位置位于屏幕像素坐标 $(x, y)$ 处,尺寸为 $w \times h$。
位图格式( \mach{format} )既可以为\mach{L1} 也可以为 \mach{L8} 。
该绘图进程将一直持续,直到用户主动调用 \nameref{cmd:stop} 指令为止。
关于 \nameref{cmd:sketch} 的具体例程请参见 \xref{sketching} 的内容。

}
\highcmd{slider}{Something}

\begin{framed}
\begin{verbatim}
void cmd_slider(int16_t x,
                int16_t y,
                uint16_t w,
                uint16_t h,
                uint16_t options,
                uint16_t val,
                uint16_t range);
\end{verbatim}
\end{framed}

\sbs{0020}{
\mach{slider} 指令用于绘制控制滑块,其位置位于屏幕坐标 $(x, y)$ 处,尺寸为 $w \times h$。
分数值( \mach{val} / \mach{range} )用于设定滑块的位置。
\aspect
\threed
}



\highcmd{spinner}{Something}

\begin{framed}
\begin{verbatim}
void cmd_spinner(int16_t x,
                 int16_t y,
                 byte style,
                 byte scale);
\end{verbatim}
\end{framed}

\vspace{20pt}
\sbs{0021}{
\mach{spinner} 指令用于绘制一个``等待'' 动画,
其位置以屏幕像素坐标 $(x, y)$ 为中心。
\mach{style} 用于设定该动画的风格;
\mach{0} 表示圆形动画,
\mach{1} 表示线形动画,
\mach{2} 表示时钟动画,
\mach{3} 表示转盘动画。
\mach{scale} 用于设定动画图片的尺寸:
$0\sim2$ 表示由小到大,


% Use \nameref{cmd:stop} to stop the spin animation.
}

\highcmd{stop}{Something}

\begin{framed}
\begin{verbatim}
void cmd_stop(void);
\end{verbatim}
\end{framed}

\mach{stop} 指令用于停止当前运行的小动画或正在执行的绘图操作。
具体参见 \nameref{cmd:spinner} 和 \nameref{cmd:sketch} 的内容。


\highcmd{text}{Something}

\begin{framed}
\begin{verbatim}
void cmd_text(int16_t x,
              int16_t y,
              byte font,
              uint16_t options,
              const char s);
\end{verbatim}
\end{framed}

\sbs{helloworld}{
\mach{text} 指令用于绘制一文本字符串 \mach{s} ,其字体由 \mach{font} 设定,
其位置位于屏幕像素坐标 $(x, y)$ 处。
\centeropts
\example{text}{helloworld}
}

\highcmd{toggle}{Something}

\begin{framed}
\begin{verbatim}
void cmd_toggle(int16_t x,
                int16_t y,
                int16_t w,
                byte font,
                uint16_t options,
                uint16_t state,
                const char *s);
\end{verbatim}
\end{framed}

\sbs{0023}{
\mach{toggle} 指令用于在屏幕像素坐标 $(x, y)$ 处绘制一个宽度为 \mach{w} 像素的拨动开关。
其拨动按钮的位置由 \mach{state} 设定;
$0$ 表示其位置在最左侧, $65535$ 表示在最右侧。
其标注文本为一对字符串,由ASCII字符码 \mach{0xff} 隔开。
\threed
}


\highcmd{track}{Something}

\begin{framed}
\begin{verbatim}
void cmd_track(int16_t x,
               int16_t y,
               uint16_t w,
               uint16_t h,
               byte tag);
\end{verbatim}
\end{framed}

\mach{track} 指令用于要求GPU追踪对于标签值为 \mach{tag} 的像素点的触摸,
并将相应值送入\mach{GD.inputs.track\_val}。
位于屏幕像素坐标 $(x, y)$ 处,尺寸为 $w \times h$ 的矩形空间用于规定追踪的区域。
当此区域仅为 $1 \times 1$ 的像素点时,输出值为触摸点相对于起始点 $(x, y)$ 的Furmans角度,
由 \mach{GD.inputs.track\_val}  存储该角度。
角度追踪功能对于 \nameref{cmd:dial} 、 \nameref{cmd:clock} 这类控件以及含有旋转控制单元的游戏程序(例如 \nameref{nightstrike} )十分实用。

除此以外, 追踪控制均是沿尺寸为 $w \times h$ 的矩形区域长轴方向进行线性追踪的。
此时 \mach{GD.inputs.track\_val} 的值表示沿矩形区域长轴上移动的距离,其取值范围为 $0\sim65535$ 。
线形追踪十分适用于 \nameref{cmd:scrollbar} 、 \nameref{cmd:toggle} 以及 \nameref{cmd:slider} 这类控件。

关于 \nameref{cmd:track} 的具体例程请参见 \nameref{widgets} 的内容。

\highcmd{translate}{Something}

\begin{framed}
\begin{verbatim}
void cmd_translate(int32_t tx,
                   int32_t ty);
\end{verbatim}
\end{framed}

\mach{translate} 指令用于对位图变换矩阵进行位移为 $(tx, ty)$ 的平移变换。 
\fsixteen

